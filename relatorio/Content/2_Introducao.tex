�\chapter{Introdu��o}
\label{Introducao}

\hspace{0,5cm}
Realmente introduz o leitor indicando quais s�o as dire��es do trabalho ? apresenta o tema e o objeto do trabalho e cont�m as Refer�ncias do Estado da arte (quem est� fazendo e em que n�vel os trabalhos da �rea est�o hoje) \cite{referencia1}.

Outra refer�ncia para a bibliografia \cite{referencia2}.

Segundo \cite{referencia3} h� uma sequ�ncia l�gica para a reda��o da monografia como apresenta em \cite{enzo}.

Refer�ncia para a figura \ref{logo}.

\begin{figure}[H]
 	\centering
 	\includegraphics[scale=0.35]{./Resources/logo_eesc_vertical.png}
 	\caption{Logo da EESC.}
 	\label{logo}
 \end{figure}



% % % % % % % % % % % % % % % % % % % % % % % % % % % % % % % % % % % % % % % % % % % % % % % % % % %
\section{Objetivos}
\hspace{0,5cm}
Objetivos do trabalho.



% % % % % % % % % % % % % % % % % % % % % % % % % % % % % % % % % % % % % % % % % % % % % % % % % % %
\section{Motiva��o (opcional)}
\hspace{0,5cm}
Descrever a motiva��o do trabalho.


% % % % % % % % % % % % % % % % % % % % % % % % % % % % % % % % % % % % % % % % % % % % % % % % % % %
\section {Justificativas/relev�ncia(opcional)}
\hspace{0,5cm}
Justificativa do trabalho.


% % % % % % % % % % % % % % % % % % % % % % % % % % % % % % % % % % % % % % % % % % % % % % % % % % %
\section {Organiza��o do Trabalho(opcional)}
\hspace{0,5cm}
Este trabalho est� distribu�do em XXX cap�tulos, incluindo esta introdu��o, dispostos conforme a descri��o que segue:

Cap�tulo 2: Descreve .....................................................................................

Cap�tulo 3: Discorre sobre .....................................................................................

Cap�tulo 4: Apresenta .....................................................................................




























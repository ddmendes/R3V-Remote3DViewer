�\chapter{Introdu��o}
\label{Introducao}

Neste documento apresenta-se uma poss�vel abordagem do desenvolvimento de um prot�tipo de sistemas distribu�dos para realidade virtual, onde transportamos o usu�rio para outro ambiente. Para isso controla-se uma c�mera remotamente atrav�s dos sensores inerciais de um \emph{smartphone} Android. O projeto se baseia no desenvolvimento de dispositivos semelhantes, como por exemplo o Oculus Rift que prop�e a cria��o de uma nova plataforma de intera��o para o usu�rio. Seguindo este conceito, uma abordagem mais simples foi proposta, utilizando-se do Cardboard da Google que constr�i, utilizando-se de uma estrutura em papel�o e o celular do usu�rio, um �culos de realidade virtual.\\


% % % % % % % % % % % % % % % % % % % % % % % % % % % % % % % % % % % % % % % % % % % % % % % % % % %
\section{Objetivos}
\hspace{0,5cm}

  A partir deste projeto pretende-se desenvolver um sistema de monitoramento em tempo real onde o usu�rio tem o controle, atrav�s de seu \emph{smartphone}, de uma c�mera IP. A aplica��o se utiliza-se de realidade aumentada, simulando a presen�a do observador no ambiente observado.\\
  O efeito de vis�o tridimensinal � dado atrav�s do algoritmo de Parallax que se baseia numa imagem 2D e constr�i duas imagens, que comp�e uma simula��o 3D.



% % % % % % % % % % % % % % % % % % % % % % % % % % % % % % % % % % % % % % % % % % % % % % % % % % %
\section {Organiza��o do Trabalho(opcional)}
\hspace{0,5cm}
Este trabalho est� distribu�do em 5 cap�tulos, incluindo esta introdu��o. Neste cap�tulo intrduziu-se o projeto a ser descrito neste documento. O segundo cap�tulo apresenta os conceitos necess�rios para o entendimento da teoria utilizada. O terceiro cap�tulo discorre sobre os materiais e m�todos necess�rios para o desenvolvimento do projetio. O quarto cap�tulo apresenta os resultados obtidos com o projeto. Por fim, no quinto cap�tulo, tiram-se conclus�es do projeto apresentado neste documento.



























